\documentclass[a4paper,12pt]{article}
\usepackage[utf8]{inputenc}
\usepackage[polish]{babel}
\usepackage{geometry}
\usepackage{graphicx}
\usepackage{listings}
\usepackage{amsmath}
\geometry{margin=1in}

\title{\textbf{Dokumentacja Projektu}}
\author{Ksawery Józefowski}
\date{\today}

\begin{document}

\maketitle

\tableofcontents
\newpage

\section{Spis użytych technologii}
W projekcie wykorzystano następujące technologie:
\begin{itemize}
    \item \textbf{Python} – do implementacji skryptów generujących dane i komunikacji z bazą danych.
    \item \textbf{MySQL} – system zarządzania relacyjną bazą danych.
    \item \textbf{mysql.connector} - biblioteka Pythona do integracji z bazą danych. 
    \item \textbf{Faker} – biblioteka Pythona do generowania danych testowych.
    \item \textbf{Unidecode} – biblioteka Pythona do obsługi znaków.
    \item \textbf{DBI} - pakiet pomagający połączyć R z systemami zarządzania bazami danych.
    \item \textbf{RMySQL} - pakiet do integracji R z MySQL.
    \item \textbf{knitR} - pakiet R do generowania Raportów.
    \item \textbf{RMarkdown} – do analizy danych i generowania raportów.
    \item \textbf{DrawSQL} - do wizualizacji schematu bazy danych.
    \item \textbf{Overleaf} - do tworzenia dokumentacji w LaTeX.
\end{itemize}

\section{Lista plików i opis ich zawartości}
\begin{itemize}
    \item \texttt{main.py} – skrypt w języku Python, odpowiedzialny za generowanie danych testowych i wypełnianie bazy danych.
    \item \texttt{schemat.sql} – plik SQL definiujący strukturę bazy danych.
    \item \texttt{Raport\_Projekt.rmd} – kod źródłowy raportu analitycznego w formacie RMarkdown.
    \item \texttt{Raport\_Projekt.pdf} – wygenerowany raport w formacie PDF.
    \item \texttt{Dokumentacja.tex} – kod źródłowy dokumentacji w LaTeX.
    \item \texttt{Dokumentacja.pdf} – wygenerowana dokumentacja w formacie PDF.
    \item \texttt{schematbazy.png} - obraz wykorzystywany przez dokumentację.
\end{itemize}

\section{Kolejność i sposób uruchamiania plików}
\begin{enumerate}
    \item Uruchomienie pliku \texttt{schemat.sql} w celu utworzenia struktury bazy danych.
    \item Wykonanie skryptu \texttt{main.py}, który wypełnia bazę danych danymi testowymi.
    \item Analiza danych za pomocą kodu w \texttt{Raport\_Projekt.rmd} i wygenerowanie raportu PDF.
\end{enumerate}

\section{Schemat projektu bazy danych}
\includegraphics[width=\textwidth]{schematbazy.png}

\newpage

\section{Lista zależności funkcyjnych z wyjaśnieniem}
\subsection{Pracownicy}
\begin{itemize}
    \item Zależności funkcyjne: \\
    \texttt{PracownikID} $\rightarrow$ \{Imie, Nazwisko, Stanowisko, DataZatrudnienia, Telefon, Email, Wynagrodzenie\}
    \item Wyjaśnienie: \\
    \texttt{PracownikID} jest kluczem głównym i identyfikuje każdego pracownika.
\end{itemize}

\subsection{Rodzaje Wycieczek}
\begin{itemize}
    \item Zależności funkcyjne: \\
    \texttt{RodzajID} $\rightarrow$ \{Nazwa, Opis\}
    \item Wyjaśnienie: \\
    \texttt{RodzajID} jest kluczem głównym tabeli i identyfikuje każdy rodzaj wycieczki.
\end{itemize}

\subsection{Kierunki Wycieczek}
\begin{itemize}
    \item Zależności funkcyjne: \\
    \texttt{KierunekID} $\rightarrow$ \{Lokalizacja, Opis\}
    \item Wyjaśnienie: \\
    \texttt{KierunekID}  jest kluczem głównym i identyfikuje lokalizacje i opis kierunku wycieczki.
\end{itemize}

\subsection{Wycieczki}
\begin{itemize}
    \item Zależności funkcyjne: \\
    \texttt{WycieczkaID} $\rightarrow$ \{RodzajID, KierunekID, Cena, Koszt\} \newline
    \texttt{RodzajID} $\rightarrow$ \{Nazwa, Opis\} \textit{(poprzez klucz obcy)} \newline
    \texttt{KierunekID} $\rightarrow$ \{Lokalizacja, Opis\} \textit{(poprzez klucz obcy)}
    \item Wyjaśnienie: \\
    \texttt{WycieczkaID} jest kluczem głównym i identyfikuje atrybuty związane z danymi wycieczkami.
\end{itemize}

\subsection{Klienci}
\begin{itemize}
    \item Zależności funkcyjne: \\
    \texttt{KlientID} $\rightarrow$ \{Imie, Nazwisko, Telefon, Email, Ulica, Miasto, KontaktAlarmowyImie, KontaktAlarmowyTelefon, KontaktAlarmowyEmail\}
    \item Wyjaśnienie: \\
    \texttt{KlientID} jest kluczem głównym i identyfikuje każdego klienta.
\end{itemize}

\subsection{Udziały Wyjazdów}
\begin{itemize}
    \item Zależności funkcyjne: \\
    \texttt{ID} $\rightarrow$ \{WycieczkaID, KlientID\} \newline
    \texttt{WycieczkaID} $\rightarrow$ \{RodzajID, KierunekID, Cena, Koszt\} \textit{(poprzez klucz obcy)} \newline
    \texttt{KlientID} $\rightarrow$ \{Imie, Nazwisko, Telefon, Email, Ulica, Miasto, KontaktAlarmowyImie, KontaktAlarmowyTelefon, KontaktAlarmowyEmail\} \textit{(poprzez klucz obcy)}
    \item Wyjaśnienie: \\
    \texttt{ID} jest kluczem głównym i identyfikuje identyfikatory wycieczki i klienta biorącego w niej udział.
\end{itemize}

\subsection{Zrealizowane Wyjazdy}
\begin{itemize}
    \item Zależności funkcyjne: \\
    \texttt{WyjazdID} $\rightarrow$ \{WycieczkaID, DataWyjazdu, DataPowrotu, PracownikID\}
    \newline
    \texttt{WycieczkaID} $\rightarrow$ \{RodzajID, KierunekID, Cena, Koszt\} \textit{(poprzez klucz obcy)} \newline
    \texttt{PracownikID} $\rightarrow$ \{Imie, Nazwisko, Stanowisko, DataZatrudnienia, Telefon, Email, Wynagrodzenie\} \textit{(poprzez klucz obcy)}
    \item Wyjaśnienie: \\
    \texttt{WyjazdID} jest kluczem głównym i identyfikuje szczegóły zrealizowanego wyjazdu.
\end{itemize}

\subsection{Wypłaty Pracowników}
\begin{itemize}
    \item Zależności funkcyjne: \\
    \texttt{WyplataID} $\rightarrow$ \{PracownikID, Kwota, DataWyplaty\} \newline
    \texttt{PracownikID} $\rightarrow$ \{Imie, Nazwisko, Stanowisko, DataZatrudnienia, Telefon, Email, Wynagrodzenie\} \textit{(poprzez klucz obcy)}
    \item Wyjaśnienie: \\
    \texttt{WyplataID} jest kluczem głównym i identyfikuje szczegóły wypłaty dla pracownika.
\end{itemize}

\newpage
\section{Uzasadnienie, że baza danych jest w postaci \textit{EKNF}}

\subsection{Analiza relacji}
\begin{itemize}
    \item Wszystkie relacje są w \textit{3NF}.
    \begin{itemize}
        \item Wszystkie atrybuty są w pełni zależne od klucza głównego. Nie występują częściowe zależności funkcyjne, ponieważ każda relacja posiada jednoznaczny klucz główny, który determinuje wszystkie pozostałe atrybuty.
        \item Nie występują tranzytywne zależności między atrybutami niekluczowymi. Jeśli istnieją powiązania między tabelami, są one reprezentowane przez klucze obce, które jednoznacznie wskazują odpowiednie dane w innych relacjach.
    \end{itemize}
    \item W każdej relacji każda zależność funkcyjna $X \rightarrow Y$ jest spełniona przez nadklucz:
    \begin{itemize}
        \item Zale\dot{z}no\'sci funkcyjne w każdej relacji odwołują się do kluczy głównych lub kluczy obcych, które jednoznacznie identyfikują wszystkie atrybuty.
        \item Nie istnieją żadne atrybuty zależne od części klucza głównego.
    \end{itemize}
\end{itemize}

\subsection{Wniosek}
Baza danych spełnia wymagania postaci normalnej (\textit{EKNF}), ponieważ każda relacja:
\begin{itemize}
    \item Nie zawiera częściowych zależności funkcyjnych.
    \item Każda zależność funkcyjna $X \rightarrow Y$ jest spełniona przez nadklucz.
\end{itemize}


\section{Najtrudniejsze elementy projektu}
\begin{itemize}
    \item Zachowanie poprawności wszystkich wprowadzanych danych.
    \item Zapewnienie unikalności danych takich jak e-maile i numery telefonów.
    \item Utrzymanie spójności bazy danych przy złożonych zależnościach.
\end{itemize}

\end{document}
